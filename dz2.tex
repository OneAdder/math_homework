\documentclass[a4paper,10pt]{article}
\usepackage[utf8]{inputenc}

\usepackage{polyglossia}
\setotherlanguage{english}
\setdefaultlanguage{russian}

\usepackage{adjustbox}
\usepackage{graphicx}

\usepackage{fontspec}
\usepackage{xunicode}
\usepackage{xltxtra}
\usepackage{libertine} 
\usepackage{indentfirst}
\usepackage{amsmath}
\usepackage{amsfonts}
\usepackage{enumitem}


%opening
\title{ДЗ 2}
\author{Михаил Воронов}

\begin{document}

\maketitle

\section{Задача 1.}
\begin{enumerate}[label=(\alph*)]
 \item $2x^3 - 3x^2 + 1 = O(x^3), x \to +\infty$
  \begin{equation*}
   \lim_{x \to +\infty} \frac{2x^3 - 3x^2 + 1}{x^3} = 
   \lim_{x \to +\infty} 2 - \frac{3}{x} + \frac{1}{x^3} = 2
  \end{equation*}
  Ответ: да, верно.
 \item $x + x^2sinx = O(x^3), x \to -\infty$
  \begin{equation*}
   \lim_{x \to -\infty} \frac{x + x^2sinx}{x^3} =
   \lim_{x \to -\infty} \Big(\frac{1}{x^2} + \frac{sinx}{x}\Big)
  \end{equation*}
  В этом пределе первая часть стремится к нулю. Числитель функции может принимать значения от $-1$ до $1$, поэтому вторая часть тоже стремится к нулю. Значит, предел равен нулю.
  Ответ: да, верно.
  \item $xsinx = o(x^3), x \to 0$
  \begin{equation*}
   \lim_{x \to 0} \frac{xsinx}{x^3} = 
   \lim_{x \to 0} \frac{sinx + xcosx}{3x^2} = 
   \lim_{x \to 0} \frac{2cosx - xsinx}{6x} = +\infty
  \end{equation*}
  Ответ: нет, неверно.
  \item $x^ne^{-x} = o(x^2), x \to +\infty$
  \begin{equation*}
   \lim_{x \to +\infty} \frac{x^ne^{-x}}{x^2} = 
   \lim_{x \to +\infty} \frac{x^{n-2}}{e^x} = 0
  \end{equation*}
  Требуется пояснение к последнему шагу: там есть неопределённость, поэтому мы применим правило Лопиталя $n - 2$ раза, тогда в числителе останется константа, а в знаменателе $e^x$, которое стремится к $+\infty$.
  Ответ: да, верно.
\end{enumerate}

\section{Задача 2.}
\begin{enumerate}[label=(\alph*)]
 \item $cos(x)$ до $x^7$
  
 $cox(x) = cos(0) + \frac{-sin0}{1!}x + \frac{-cos0}{2!}x^2
 + \frac{sin0}{3!}x^3 + \frac{cos0}{4!}x^4 + \frac{-sin0}{5!}x^5
 + \frac{-cos0}{6!}x^6 + $
 
 $ + \frac{sin0}{7!}x^7 = 1 - \frac{x^2}{2!} + \frac{x^4}{4!}
 - \frac{x^6}{6!} + ... $
  
 \item $e^{2x - x^{2}}$ до $x^3$
 
 $\big(e^{2x - x^{2}}\big)' = 2(1 - x)e^{2x - x^{2}}$
 
 $\big(e^{2x - x^{2}}\big)'' = e^{2x - x^{2}}(2 - 2x)
 (2 - 2x) - 2e^{2x - x^{2}} = 2(2(1-x)^2 - 1)e^{2x - x^{2}} $
 
 $\big(e^{2x - x^{2}}\big)''' = \big(2 - 8x + 4x^2\big)'
 e^{2x - x^{2}} + (2 - 8x + 4x^2)(2x - 2)e^{2x - x^{2}} = $
 
 $ = (-8x^3 + 24x^2 - 12x - 4)e^{2x - x^{2}} $
 
 $ e^{2x - x^{2}} = 1 + \frac{2}{1!}x + \frac{2}{2!}x^2 + \frac{0}{3!}x^3 + ... = 
 1 + 2x + x^2 + ... $
 
 \item $cos(e^{2x - x^{2}})$ до $x^3$
 
 $ cos(e^{2x - x^{2}}) \big[$ до $x^3\big] = 1 - \frac{(x^2 + 2x + 1)^2}{2} $ 
\end{enumerate}

\section{Задача 3.}
\begin{enumerate}[label=(\alph*)]
 \item
 $$\lim_{x \to 0} \frac{cosx - 1}{x^2}x = \lim_{x \to 0} \frac{-\frac{x^2}{2!}}{x^2}x
 = \lim_{x \to 0}-0,5x = 0 $$
 \item
 $$ \lim_{x \to 0} \frac{e^{2x - x^{2}}-1-2x+3x^2}{x^2} = \lim_{x \to 0} \frac{1 + 2x + x^2 - 1 - 2x + 3x^2}{x^2}  = 4 $$
 
 \item
 $$ \lim_{x \to 0} cos(e^{2x - x^{2}}) = cos1 $$ Я бы рад здесь что-то пораскладывать, но неопределённости нет
 
 \item
 $$ \lim_{x \to 0} \frac{e^xsinx - x(1+x)}{x^3} = \lim_{x \to 0}
 \frac{(1+x)(x-\frac{x^3}{3!}) - x(1+x)}{x^3}
 = \lim_{x \to 0} \frac{-\frac{x^3}{6} (1+x)}{x^3} = $$
 
 $$ = \lim_{x \to 0} -\frac{1+x}{6} = -\frac{1}{6}$$.
\end{enumerate}

\section{Задача 4.}

$$y = ln(1+7x)$$

$$y' = \frac{7}{1+7x}$$

$$y'' = \frac{-49}{(1+7x)^2}$$

$$y''' = \frac{646}{(1+7x)^3}$$

$$ y = ln(1+7x) = ln(22) + \frac{7(x-3)}{22} - \frac{49(x-3)^2}{968} 
+ \frac{686(x-3)^3}{63 888} - ... $$

\section{Задача 5.}

\begin{enumerate}[label=(\alph*)]
 \item
 $$y = ln(1+x)$$

 $$y' = \frac{1}{x+1}$$

 $$y'' = -\frac{1}{(x+1)^2} $$

 $$y''' = \frac{2}{(x+1)^3} $$
 
 $$y'''' = -\frac{6}{(x+1)^4}  $$
 
 $$ ln(1+x) = ln(1) + \frac{x}{1!} - \frac{x^2}{2!} + \frac{2x^3}{3!}
 - \frac{6x^4}{4!} = x - \frac{x^2}{2} + \frac{x^3}{3} - \frac{x^4}{4} + ...$$
 
 $$ \sum^{\infty}_{n=1} \frac{x^n}{n} (-1)^{n+1} $$
 
 Попробуем доказать, что это действительно так. Каждая производная от функции $ln(x+1)$ содержит выражение $x+1$ в отрицательной степени, значит, знак с каждой новой производной будет меняться. Остаётся понять, куда уходит факториал. С каждой новой производной мы умножаем значения всех степеней предыдущих производных, то есть берём факториал $n-1$. Так как $\frac{n!}{(n-1)!} = n$, в знаменателе остаётся $n$.
 \item
 $$ ln(1,1) = ln(1+0,1) \approx \frac{1}{10} - \frac{1}{200} + \frac{1}{3000}
 - \frac{1}{40 000} + \frac{1}{500 000} = $$

 $$ = \textbf{0.09531}0(3) $$
\end{enumerate}
\end{document}
