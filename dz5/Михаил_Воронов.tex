\documentclass[a4paper,10pt]{article}
\usepackage[utf8]{inputenc}

\usepackage{polyglossia}
\setotherlanguage{english}
\setdefaultlanguage{russian}

\usepackage{adjustbox}
\usepackage{graphicx}

\usepackage{fontspec}
\usepackage{xunicode}
\usepackage{xltxtra}
\usepackage{libertine} 
\usepackage{indentfirst}
\usepackage{amsmath}
\usepackage{amsfonts}
\usepackage{enumitem}


%opening
\title{ДЗ 5}
\author{Михаил Воронов}

\begin{document}

\maketitle

\section{}
Подумаем, в какой доле случаев сумма результатов бросков будет чётной.
При первом броске вероятность получить чётное число равна $\frac{1}{2}$
(половина всех чисел на кубике чётна).

Сумма двух чётных чисел чётна. Если на первом броске число чётное,
то, чтобы получилась чётная сумма, на втором броске должно быть
чётное число. Это произойдёт в половине случаев.

Сумма двух нечётных чисел чётна. Если на первом броске число нечётное,
то, чтобы получилась чётная сумма, на втором броске должно быть
нечётное число. Это произойдёт в половине случаев.

Итак, результат будет чётным в половине случаев, а значит:
$$ P(X=0) = \frac{1}{2} $$
$$ P(X=1) = \frac{1}{2} $$

Произведение будет нечётным, когда резульатт обоих бросков нечётный.
Для каждого числа из первого броска может быть три варианта нечётных
результатов из второго броска. То есть $3 \times 3 = 9$.
Всего может быть $6 \times 6 = 36$ исходов. Значит, число будет нечётным
в $\frac{1}{4}$ случаев. Итак:
$$ P(Y=0) = \frac{3}{4} $$
$$ P(Y=1) = \frac{1}{4} $$

Заметим, что нечётная сумма возникает только при сложение чётного с нечётным и наоборот.
Произведение чётного с нечётным всегда чётно. А значит:
$$ P(X=1, Y=0) = \frac{1}{2} $$
$$ P(X=1, Y=1) = 0 $$

Дальше заметим, что чётная сумма возникает при двух чётных или двух нечётных.
Произведение двух чисел чётно, произведение двух нечётных нечётно.
Значит каждая из этих ситуаций занимает половину половины исходов:
$$ P(X=0, Y=0) = \frac{1}{4} $$
$$ P(X=0, Y=1) = \frac{1}{4} $$

\section{}
Так как не даны цифры на гранях, пронумеруем их как $[0, 1, 2]$.
Дальше лучше не думать, а рисовать.

\begin{center}
\begin{tabular}{|c|c|c|c|}
\hline
$\xi$ & \textbf{0} & \textbf{1} & \textbf{2} \\
\hline
\textbf{0} & 0 & 1 & 2 \\
\hline
\textbf{1} & 1 & 2 & 3 \\
\hline
\textbf{2} & 2 & 3 & 4 \\
\hline
\end{tabular}
\end{center}

\begin{center}
\begin{tabular}{|c|c|c|c|}
\hline
$\eta$ & \textbf{0} & \textbf{1} & \textbf{2} \\
\hline
\textbf{0} & 0 & 1 & 2 \\
\hline
\textbf{1} & -1 & 0 & 1 \\
\hline
\textbf{2} & -2 & -1 & 0 \\
\hline
\end{tabular}
\end{center}

\subsection{}
Сгруппируем в одну таблицу:

\begin{center}
\begin{tabular}{|c|c|c|c|c|c|}
\hline
$\eta$ \vline $\xi$ & \textbf{0} & \textbf{1} & \textbf{2} & \textbf{3} & \textbf{4} \\
\hline
\textbf{-2} & 0 & 0 & 1/9 & 0 & 0 \\
\hline
\textbf{-1} & 0 & 1/9 & 0 & 1/9 & 0 \\
\hline
\textbf{0} & 1/9 & 0 & 1/9 & 0 & 1/9 \\
\hline
\textbf{1} & 0 & 1/9 & 0 & 1/9 & 0 \\
\hline
\textbf{2} & 0 & 0 & 1/9 & 0 & 0 \\
\hline
\end{tabular}
\end{center}

\subsection{}

\begin{center}
\begin{tabular}{|c|c|c|c|c|c|}
\hline
$\xi$ & \textbf{0} & \textbf{1} & \textbf{2} & \textbf{3} & \textbf{4} \\
\hline
P & 1/9 & 2/9 & 1/3 & 2/9 & 1/9 \\
\hline
\end{tabular}
\end{center}

\begin{center}
\begin{tabular}{|c|c|c|c|c|c|}
\hline
$\eta$ & \textbf{-2} & \textbf{-1} & \textbf{0} & \textbf{1} & \textbf{2} \\
\hline
P & 1/9 & 2/9 & 1/3 & 2/9 & 1/9 \\
\hline
\end{tabular}
\end{center}

\subsection{}
$$
\mathbb{E}\xi = 0 \times \frac{1}{9} + 1 \times \frac{2}{9} + 2 \times \frac{1}{3}
+ 3 \times \frac{2}{9} + 4 \times \frac{1}{9} = 2
$$
$$
\mathbb{D}\xi = 0^2 \times \frac{1}{9} + ... + 4^2 \times \frac{1}{9} - 2^2 = \frac{1}{3}
$$
$$\sigma\xi = \sqrt{\mathbb{D}\xi} = \frac{1}{\sqrt{3}}$$

$$
\mathbb{E}\eta = -2 \times \frac{1}{9} - 1 \times \frac{2}{9} + 0 \times \frac{1}{3}
+ 1 \times \frac{2}{9} + 2 \times \frac{1}{9} = 0
$$
$$
\mathbb{D}\eta = (-2)^2 \times \frac{1}{9} + ... + 2^2 \times \frac{1}{9} - 0 = \frac{10}{9}
$$
$$\sigma\eta = \frac{\sqrt{10}}{3}$$

\subsection{}
$$ \mathbb{E}(\xi\eta) = -4/9 - 1/9 - 3/9 + 1/9 + 3/9 + 4/9 = 0 $$
$$ Cov(\xi\eta) = (\mathbb{E}\xi\eta) - \mathbb{E}\xi \times \mathbb{E}\eta = 0 $$

$$ \rho(\xi\eta) = \frac{Cov(\xi\eta)}{\sigma\xi \times \sigma\eta} = 0 $$

\section{}
Испытаний много, поэтому воспользуемся интегральной теоремой Лапласа.

$n = 40 000$

$p = 0,02$

$P \approx \Phi\Bigg(\frac{840 - 40 000 \times 0,02}{\sqrt{40 000 \times 0,02 \times 0,98}}\Bigg) - \Phi\Bigg(\frac{828 - 40 000 \times 0,02}{\sqrt{40 000 \times 0,02 \times 0,98}}\Bigg) = $

$= \Phi(1,43) - \Phi(1) \approx 0,4236 - 0,3413 = 0,0823$

\section{}
В принципе, здесь тоже следует применить теорему Лапласа. Только здесь придётся подумать.

Заметим, что, раз люди заходят парами, то можно от них избавиться и считать,
что цирк вмещает 500 пар.

$n = 500$

$p = \frac{1}{2}$

Обозначим за $x$ искомое количество мест. Значит, нам требуется узнать, от нуля до какого количества
пар может придти так, что они поместятся в свой гардероб при данной вероятности. То есть:

$$
\Phi\Bigg(\frac{x - 500 \times \frac{1}{2}}{\sqrt{500 \times \frac{1}{4}}}\Bigg)
- \Phi\Bigg(\frac{-500 \times \frac{1}{2}}{\sqrt{500 \times \frac{1}{4}}}\Bigg)
= 95\%
$$

$$
\Phi\Bigg(\frac{-500 \times \frac{1}{2}}{\sqrt{500 \times \frac{1}{4}}}\Bigg)
\approx -\Phi\big(22,3607\big) \approx -0,5
$$

$$
\Phi\Bigg(\frac{x - 500 \times \frac{1}{2}}{\sqrt{500 \times \frac{1}{4}}}\Bigg)
\approx 0,45
$$

$$
\frac{x - 500 \times \frac{1}{2}}{\sqrt{500 \times \frac{1}{4}}} \approx 1,65
$$

$$
\frac{x}{11,1803} - 22,3607 \approx 1,65
$$

$$ x \approx 268.4468 $$

Ответ: значит, требуется всего 537 мест в каждом гардеробе, чтобы вероятность неудачи
была меньше $\frac{1}{50}$.

\section{}
\subsection{}
Страховая компания "Ну заходи, голубчик" окажется в убытке, если каждый $\frac{100 000}{1200} = 83$
контракт выльется в страховой случай. То есть, если страховых случаев будет больше чем
$\frac{10 000}{83} = 120$. То есть снова Лаплас.

$n = 10 000$

$p = 0,006$

$$
\Phi\Bigg(\frac{10 000 - 10 000 \times 0,006}{\sqrt{10 000 \times 0,006 \times 0,994}}\Bigg)
- \Phi\Bigg(\frac{121 - 10 000 \times 0,006}{\sqrt{10 000 \times 0,006 \times 0,994}}\Bigg) =
$$
$$
\approx \Phi\big(1287\big) - \Phi\big(7,7227\big) \approx 0
$$

То есть, компания примерно никогда не будет в убытке. Однако, хотелось бы получить какой-то
более внятный ответ, тем более что мы знаем, что вероятность получилась больше нуля.
Для этого воспользуемся следующим кодом на Питоне:

\texttt{>>> from scipy.stats import norm}

\texttt{>>> laplace\_func = lambda x: norm.cdf(x) - 0.5}

\texttt{>>> P = laplace\_func(1287) - laplace\_func(7.7227)}

\texttt{>>> print('\%0.40f' \% P)}

\texttt{0.0000000000000056621374255882983561605215}


\subsection{}
Для прибыли в $4 000 000Я$ достаточно, чтобы $80$ контрактов не вылились в страховой случай.

$$
\Phi\Bigg(\frac{80 - 10 000 \times 0,006}{\sqrt{10 000 \times 0,006 \times 0,994}}\Bigg)
- \Phi\Bigg(\frac{0 - 10 000 \times 0,006}{\sqrt{10 000 \times 0,006 \times 0,994}}\Bigg) =
$$
$$
\approx \Phi\big(2,59\big) + \Phi\big(1287\big) \approx 0,495 + 0,5 = 0,995
$$

\end{document}
