\documentclass[a4paper,10pt]{article}
\usepackage[utf8]{inputenc}
\usepackage{ebgaramond}
\usepackage{polyglossia}
\usepackage{indentfirst}
\setdefaultlanguage{russian}
\setotherlanguage{english}
\usepackage{amsmath}
\usepackage{unicode-math}

%opening
\title{Домашнее задание 4}
\author{Михаил Воронов}

\begin{document}

\maketitle

\section{}

Итак, посмотрим, какие вероятности у нас имеются:
\begin{itemize}
 \item $P(\Delta) = 0,174$
 \item $P(\text{Я}|\Delta) = 0,035$
 \item $P(\Delta|\text{Я}) = 0,701$
 \item $P(\text{П}|\text{Я}) = 0,001$
 \item $P(\text{ОН}|\Delta\text{ЯП}) = 1$
\end{itemize}

\subsection{}
Найти: $P(\Delta|\text{Я}|\Delta)$

Вероятность появления пробела равна $0,174$, верояность, что после него будет \textit{я} равна $0,035$, вероятность, что после \textit{я} будет пробел равна $0,701$. Соответственно, нам нужна ситуация, когда все три условия выполняются, то есть:
$$ 0,174 \times 0,035 \times 0,701 = 0,00426909 $$

На всякий случай, сверим наш результат с поиском по корпусу. Слово \textit{я} в \textsc{NOM.SG}, с маленькой буквы, перед и после пробела в корпусе объёмом $288 727 494$ (НКРЯ) встречается $1 051 117$ раз. То есть вероятность встретить такое слово будет $\approx 0.00364052$, что совпадает с нашим результатом с точностью до третьего знака после запятой.

Итак ответ: $0,00426909$.

\subsection{}
В общем, ситуация такая же, только символов больше:

\begin{multline*}
P(\Delta\text{ЯПОН})
= P(\Delta) \times P(\text{Я}|\Delta) \times P(\text{П}|\text{Я})
\times P(\text{ОН}|\Delta\text{ОН}) = \\
= 0,174 \times 0,035 \times 0,001 \times 1
= 0.00000609
\end{multline*}

Здесь результаты с корпусом расходятся: $\approx 0.00005890$.

Ответ: $0.00000609$

\section{}
Вероятность таракана: $0,2$. Значит, вероятность того, что таракан не попадётся ни разу за $n$ дней будет равна $(1 - 0,2)^n = 0,8^n$. Итак, вероятность того, что за $n$ дней попадётся таракан равна $1 - 0,8^n$. Мы знаем, что это всё должно быть не меньше $0,9$, то есть:
$$ 1 - 0,8^n \geqslant 0,9$$
$$ 0,8^n \leqslant 0,1 $$
Итак, чтобы найти результат, нам нужно посчитать такой логарифм:
$ \log_{0,8} 0,1 \approx 10.32 $. Значит, вероятность на десятом дне будет $<0,9$,
а на одиннадцатом больше.

Ответ: $11$.

\section{}
Первый ковбой выберет чужое лассо с вероятность $\frac{2}{3}$, второму останется выбрать из двух, то есть $\frac{1}{2}$, остаётся одно лассо, то есть третий возьмёт чужое лассо в любом случае. Итак:
$$ \frac{2}{3} \times \frac{1}{2} \times 1 = \frac{2}{6} = \frac{1}{3} $$

\section{}
У нас есть 15 семинаров и 8 потенциальных докладчиков. Попробуем разобраться, что может случиться на каждом из семинаров:
\begin{itemize}
 \item докладчиком будет сестра, $P = \frac{3}{8}$
 \item докладчиком будет дядя Ваня, $P = \frac{1}{2}$
 \item докладчиком будет руководитель, $P = \frac{1}{8}$
\end{itemize}

Представим, что происходящее -- это серия экспериментов с тремя исходами, чтобы посчитать биномиальное распределение. Время вспомнить формулу вероятности, так как у нас не два исхода, а три, то мы не сможем использовать формулу Бернулли. Хорошо бы её обобщить, чтобы она выглядела как-то так:
$$ P = x \times p_1^{k1} p_2^{k2} p_3^{k3} $$

Осталось придумать, чему здесь будет равен $x$. В формуле бернулли первый коэффициент показывает,
сколькими способами можно выбрать события ($k$) из $n$ экспериментов.
В случае формулы, у нас есть события успеха и провала, у нас событий больше.
Значит, нам надо сначала выбрать первые $k$ событий. Дальше дней станет меньше,
поэтому нам нужно будет выбрать вторые $k$ событий из $n-k^1$ экспериментов, и так далее.
То есть:
$$x = C_n^{k_{1}} C_{n-k_1}^{k_{2}} ... $$

Подставим наши данные:
$$
P = \frac{15!}{5! 8! 2!} \bigg(\frac{3}{8}\bigg)^5
\bigg(\frac{1}{2}\bigg)^8 \bigg(\frac{1}{8}\bigg)^2
\approx 0.061165
$$

\section{}
\begin{itemize}
 \item A -- подготовленный студент \textbf{не} знает ответ
 \item B -- ответ правильный
\end{itemize}
Из условия:
\begin{itemize}
 \item $P(A) = \frac{1}{10}$
 \item $P(\overline{A}) = \frac{9}{10}$
 \item $P(B|A) = \frac{1}{6}$
 \item $P(B|\overline{A}) = 1$
\end{itemize}

$$
P(B) = P(B|A)P(A) + P(B|\overline{A})P(\overline{A})
= \frac{1}{6} \times \frac{1}{10} + 1 \times \frac{9}{10}
= \frac{55}{60}
$$
$$ P(A|B) = \frac{\frac{1}{6} \times \frac{1}{10}}{\frac{55}{60}} = \frac{1}{55} $$

\section{}
\begin{itemize}
 \item B -- капибара действительно больна
 \item BB -- тест показал, что капибара больна
 \item Z -- капибара здорова
\end{itemize}
Из условия:
\begin{itemize}
 \item $P(BB|B) = 0,7$
 \item $P(BB|Z) = 0,3$
 \item $P(B) = 0,02$
 \item $P(Z) = 0,98$
 \item $P(B|BB) = ?$
\end{itemize}

Посчитаем:
$$ P(BB) = 0,7 \times 0,02 + 0,3 \times 0,98 = 0,308 $$

Дальше по формуле Байеса:
$$
P(B|BB) = \frac{P(BB|B)P(B)}{P(BB)}
= \frac{0,7 \times 0,02}{0,308} = 0,0(45)
$$

\section{}
Для вынесения решения нужно два голоса из трёх, то есть достаточно, чтобы двое присяжных вынесли одинаковое решение. Рассмотрим все ситуации, когда получается правильное решение:
\begin{itemize}
 \item $0,3 \times 0,9 \times 0,5 = 0,135$
 \item $0,7 \times 0,1 \times 0,5 = 0,035$
 \item $0,7 \times 0,9 \times 0,5 = 0,315$
 \item $0,7 \times 0,9 \times 0,5 = 0,315$
\end{itemize}

Так как любая из этих ситуаций является достаточной для принятия правильного решения, вероятности следует сложить, итоговая вероятность принятия верного решения будет равна $0,8$.

В случае, когда третий будет копировать первого, останется только две ситуации, когда будет принято правильное решение: если все примут правильное решение и если второй ошибётся. Так как мы знаем, что, если первый проголосует правильно, то и третий тоже, вероятность у третьего всегда будет один. То есть:
\begin{itemize}
 \item $0,7 \times 0,9 \times 1 = 0,63$
 \item $0,7 \times 0,1 \times 1 = 0.07$
\end{itemize}

Сумма: $0,7$.

Итак, результат изменится на одну десятую, если третий начнёт копировать председателя.

\end{document}
